\chapter{Background}\label{ch:background}

%Every semester, students ask their supervisor how to write their thesis,
%what the requirements are, and what to write in it.  
%This document tries to answer all such questions.

\section{User Modelling}

\begin{itemize}
  \item Discuss user modelling here
  \item Refer to papers and quote
\ldots
\end{itemize}

Why do people use Facebook?
I would like to cite Bob~\cite{nadkarni2012people} who has died for no reason.

Four approaches to user modelling—a qualitative research interview study of HCI professionals' practice
I would like to cite Bob~\cite{clemmensen2004four} who has died for no reason.

MySpace and Facebook: Identifying dimensions of uses and gratifications for friend networking sites
I would like to cite Bob~\cite{bonds2010myspace} who has died for no reason.

Semantic modelling of user interests based on cross-folksonomy analysis
I would like to cite Bob~\cite{szomszor2008semantic} who has died for no reason.

\section{Ranking Algorithms}

A ranking algorithm will give each item a score and order them with the item with the highest score at the top and the item with the lowest score at the bottom. The score of an item will depend on a set of criteria that the ranking algorithm uses. Social networking services will use these ranking algorithms in order to provide the user with items that will interest them. Since we are focusing on Facebook, the users will be provided with a feed that contains a lot of posts that they will receive. In Li~\cite{LiTiaLee2010} paper, they discovered that there are three major factors that could affect how interesting a user may find a particular item. They are:


\begin{itemize}
 \item Topical Preference
 \item Topological Locality
 \item Social Influence
\end{itemize}

Topical reference is the idea that most users are interested in a limited range of topics. Topological locality refers to the fact that users are interested in the topics that their friends like and Social influence basically says that users are interested in famous people such as singers or actors. These factors do provide some insight on what a user likes but could be further generalised to topics that a user likes. The Topological locality does raise an interesting notion, that is, users are more likely to like a post that a friend likes. We can summarize these two into a more general categorization of Topic classification and connections. Topic classification will be classifying the topics that a user may like while connections will be a measure of how 'close' the user is with their friend based on their interactivity. 

Topic classification is quite difficult we have to generalise a topic that they may like based on the posts that they receive. In a paper by Bur~\cite{Bur2013}, they analysed twitter tweets and tried to generalise a topic based on the tweets each user received. They have analysed two types of methods. They are:

\begin{itemize}
 \item BestOverlap
 \item UserInfoBigram
\end{itemize}

The BestOverlap method attempts to gather a huge amount of tweets and look at the common words in those tweets. A topic can be generated by the word is overlapped the most across all the tweets. In regards to our algorithm where we have to look at Facebook posts, the likelihood of word overlaps across a large amount of posts is quite low. This method would not be appropriate for our purposes.

The UserInfoBigram analyses the optional text that is provided in every tweet and generalises a topic from those words. In Facebook, almost no one uses the optional text so this method will also not work.

In order to do topic classificication we had to look to (insert your related work richard!! and explain)

Aga~\cite{Aga2014} discusses activity ranking for LinkedIn which is also a Social Networking Service. They discover two more factors that have a huge impact on whether the activity is deemed interesting or not. They used the measurement of CTR or click-through-rate which is the probability of a user clicking on the link to measure the appeal of an activity. An activity that was old had a low CTR compared to an activity that was new. It seems that the freshness of an activity or the time that the activity was made had to be taken into account in the ranking algorithm. Another factor was diversity. A huge drop in CTR was found when they gave users a repeated type of activity in their feed. We can surmise that freshness and diversity are key factors that must be considered in our ranking algorithm. The method that they have used to deal with these two issues involved re-ranking the feed with a decay factor to account for time and adding a negative score to activities of the same type. For our algorithm, we plan to utilise the same methods proposed as they have been successful.

Aga~\cite{Aga2014} also reinforces the idea that people are interested in what their friends like when they analysed the activities of co-workers and colleagues. Like Li~\cite{LiTiaLee2010}, they found that there was an increase in the click-through-rate of activities they were made by co-workers in the same organisation and colleagues. This emphasizes the importance of the factor of connections.
