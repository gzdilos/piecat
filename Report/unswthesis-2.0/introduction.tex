\chapter{Introduction}\label{ch:intro}

\begin{itemize}
  \item Discuss what an SNS is
  \item Talk About Facebook and what is a feed
  \item Talk about ranking algorithms
\ldots
\end{itemize}

Social Networking Services or SNS are platforms where a diverse range of users can share their interests, do activities together and troll their friends. In almost all social networking services users are presented with a feed. This feed is a list of items that a user may or may not like. It is presented differently depending on the SNS and is usually the first thing shown to the user as soon as he/she logs in. This feed usually contains a large amount of items needs to be ordered in some way so that the items that relate to him or items that he would enjoy would be shown in the upper half of the feed. There are many SNS out there but in this paper we will be focusing on Facebook. The reason for this is that Facebook is currently the most used SNS and incorporates many types of users. Each user will have a different want and need. We would expect that each user would want to have a personalized ranking for their own needs.



%Chapter~\ref{ch:background} explains the background for this document.
%Chapter~\ref{ch:style} states the style and submission related requirements
%to theses submitted at the school.
%Chapter~\ref{ch:content} explains content related requirements to theses.
%Chapter~\ref{ch:eval} evaluates the thesis requirements template.  Finally,
%Chapter~\ref{ch:conclusion} draws up conclusions and suggest ways to
%further improve the thesis requirements template.

